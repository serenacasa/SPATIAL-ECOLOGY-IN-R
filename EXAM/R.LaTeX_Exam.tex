\documentclass{beamer}
\usepackage{graphicx} 
\usepackage{tikz}
\usepackage{listings}
\usepackage{xcolor}
\usepackage{booktabs}
\usepackage{caption}

\usetheme{CambridgeUS}
\usecolortheme{spruce}

\title{Camp Fire Reforestation Analysis}
\subtitle{Paradise, California (2018–2025)}

\author{Serena Casagrande} 
\institute{M. n°0001138070}
\date{February 11, 2026}

\titlegraphic{
\includegraphics[width = 0.3\textwidth]{Images/Camp_Fire_2018.jpg}}

\begin{document}

\maketitle

%-----------------------------------------------------------
\section{Introduction}

\begin{frame}{Camp Fire — November 8, 2018}
The remote sensing project "Camp Fire Reforestation Analysis" employed Sentinel-2 multitemporal imagery to analyze vegetation loss and recovery following the Camp Fire (November 8, 2018) in Paradise (Butte County), described as the deadliest and most destructive wildfire in California history.

\bigskip

From November 8, 2018, over 17 days, the Camp Fire burned more than 153,000 acres (619.2 km²), killed 85 people, and destroyed over 18,000 structures in Butte County, California, devastating the town of Paradise.
\end{frame}

\begin{frame}{Aim of the Project}
This analysis investigated the post-fire vegetation recovery (in the form of NDVI changes) in the Camp Fire area, from July 2018 (pre-fire condition) until July 2025 (6.5 years post-fire), the latest peak growing season available (determining the current condition). 
\bigskip

The aims were to study:
\begin{enumerate}
    \item \textbf{The Fire Impact} \\
    Quantify immediate vegetation loss from the fire (December 2018).\\
    
    \item \textbf{Recovery Trajectory} \\
    Map post-fire NDVI recovery over 6.5 years.\\
 
    \item \textbf{Severity × Recovery Relationship} \\
     Assess the relationship between burn severity and vegetation regrowth.\\
 
    \item \textbf{Spatial Heterogeneity} \\
    Using the focal standard deviation (3×3, 5×5, 7×7)\\
\end{enumerate}
\end{frame}

%-----------------------------------------------------------
\section{Material and Methods}

\begin{frame}{Material and Methods}
\begin{columns}
    \begin{column}{0.8\textwidth}
        \centering
        \textbf{Data Collection}
        \begin{itemize}
            \item Sentinel-2 L2A — Copernicus Browser
            \item 8 cloud-free images (2018–2025)
            \item Bands: B02, B03, B04, B08 (10m resolution)
        \end{itemize}

        \bigskip
        \textbf{R Libraries:} 
        \begin{itemize}
            \item \texttt{library(terra)} 
            \item \texttt{library(viridis)} 
            \item \texttt{library(fields)} 
            \item \texttt{library(imageRy)}
        \end{itemize}
        
        \bigskip
        \textbf{Fire Perimeter:}
        \begin{itemize}
            \item Downloaded from the MTBS database
            \item All analyses were performed within the fire perimeter
        \end{itemize}
    \end{column}

\end{columns}
\end{frame}

\begin{frame}[fragile]{Custom Functions for Data Processing}
\textbf Functions were created to simply the workflow:

\bigskip
\textbf{1. Band Extraction:}

\lstset{
    language = R,
    basicstyle = \scriptsize\ttfamily,
    keywordstyle = \color{blue},
    commentstyle = \color{gray},
    stringstyle = \color{red},
    showstringspaces = false,
    breaklines = true
}

\begin{lstlisting}
s2_bands <- function(folder_name) {
  all_files <- list.files(folder_name, pattern = "\\.jp2$", 
                         recursive = TRUE, full.names = TRUE)
  band_files <- all_files[grep("B0[2348]_10m\\.jp2$", all_files)]
  rast(sort(band_files))
}
\end{lstlisting}

\bigskip
\textbf{2. Merge \& Crop:}

\begin{lstlisting}
merge_and_crop <- function(tile_north, tile_south, bbox) {
  mosaic(crop(tile_north, bbox), crop(tile_south, bbox), 
         fun = "mean")
}
\end{lstlisting}
\end{frame}

\begin{frame}[fragile]{NDVI Calculation}
\textbf{Normalized Difference Vegetation Index:}

\begin{equation}
\text{NDVI} = \frac{\text{NIR} - \text{Red}}{\text{NIR} + \text{Red}}
\end{equation}

\bigskip
For the calculation of the NDVI a function was used:

\begin{lstlisting}
calc_ndvi <- function(img) { 
  (img[[4]] - img[[3]]) / (img[[4]] + img[[3]]) 
}
\end{lstlisting}

\bigskip
where:
\begin{itemize}
    \item \texttt{img[[4]]} = NIR = Band 8 (842 nm)
    \item \texttt{img[[3]]} = Red = Band 4 (665 nm)
\end{itemize}
\end{frame}

%-----------------------------------------------------------
\section{Fire Impact Assessment}

\begin{frame}{Burn Severity Assessment — dNDVI}
The Burn severity was assessed through the calculation of the dNDVI (difference NDVI), defined as 

\begin{equation}
\text{dNDVI} = \text{NDVI}_{\text{Nov 6, 2018}} - \text{NDVI}_{\text{Dec 6, 2018}}
\end{equation}

\begin{itemize}
    \item Positive values = vegetation loss
    \item Negative values = vegetation increase
\end{itemize}

\bigskip
\textbf{Results:}
\begin{block}{Mean dNDVI = 0.143}
\end{block}

\bigskip
\textbf{Color Palette used for the plot:}
\begin{itemize}
    \item \textcolor{red}{Red} = Severe burns (dNDVI $>$ 0.35)
    \item \textcolor{yellow!80!black}{Yellow} = Moderate (0.15–0.35)
    \item \textcolor{blue}{Blue} = Unaffected ($<$ 0.15)
    \item Limited Range: [-0.5, 0.5] to avoid palette compression
\end{itemize}
\end{frame}

\begin{frame}{Fire Impact — Visual Comparison}
\centering
The extent of vegetation loss becomes evident once the Nov.2018 NDVI and Dec.2018 are plotted together with the dNDVI:

\bigskip
\includegraphics[width=\textwidth]{Images/Camp_fire_impact.png}
\captionof{figure}{\footnotesize Pre-fire NDVI (Nov 2018), Post-fire NDVI (Dec 2018), and dNDVI (fire impact)}
\end{frame}

%-----------------------------------------------------------
\section{Recovery Trajectory}

\begin{frame}{NDVI Recovery Timeline}
\centering
\textbf The analysis of the investigated July NDVIs  shows clear trajectory with three phases:

\bigskip
\begin{table}
\scriptsize
\begin{tabular}{lcccl}
\toprule
\textbf{Date} & \textbf{NDVI} & \textbf{$\Delta$ Baseline} & \textbf{Rate (yr$^{-1}$)} & \textbf{Phase} \\
\midrule
Jul 2018 & 0.394 & --- & --- & Pre-fire baseline \\
Nov 2018 & 0.352 & -10.7\% & --- & Natural decline \\
\textbf{Dec 2018} & \textbf{0.208} & \textbf{-47.2\%} & \textbf{---} & \textbf{Max. Fire impact} \\
Jul 2019 & 0.258 & -34.5\% & +0.075 & First growing season \\
Jul 2020 & 0.302 & -23.4\% & +0.044 & Early recovery \\
Jul 2022 & 0.312 & -20.8\% & +0.005 & Mid-term recovery \\
Jul 2024 & 0.344 & -12.7\% & +0.016 & Late recovery \\
\textbf{Jul 2025} & \textbf{0.342} & \textbf-13.2\% & \textbf-0.002 & \textbf Current condition \\
\bottomrule
\end{tabular}
\end{table}

\bigskip
\textbf{Key Findings:}
\begin{itemize}
    \item The Camp Fire caused a \textbf{47.2\%} NDVI loss respect to July 2018
    \item\textbf{64\%} of NDVI was restored, but still 13\% below baseline
    \item Rate: +0.075/yr (July 2019) $\rightarrow$ -0.002/yr  (July 2025)
\end{itemize}
\end{frame}

\begin{frame}{Recovery Trajectory — Graphical View}
\centering
\includegraphics[width = 0.8\textwidth]{Images/Recovery_trajectory_line.png}
\captionof{figure}{\footnotesize Mean NDVI timeline showing sharp post-fire decline and progressive recovery}
\end{frame}

\begin{frame}{Temporal NDVI Maps (2018–2025)}
\centering
Progressive NDVI recovery visible across the years.
\includegraphics[width = 0.8\textwidth]{Images/NDVIs_Temporal_Series.png}
\captionof{figure}{\footnotesize NDVI maps at peak growing season (July) from 2018 to 2025}
\end{frame}

%-----------------------------------------------------------
\section{Severity × Recovery Analysis}

\begin{frame}{Classification Using dNDVI and Recovery}
\centering
Classification performed using quantile-based thresholds:

\bigskip
\begin{columns}
\begin{column}{0.48\textwidth}
\textbf{Severity (dNDVI):}
\begin{itemize}
    \item Unburned ($<$ 0)
    \item Low (0–0.15)
    \item Moderate (0.15–0.35)
    \item High ($>$ 0.35)
\end{itemize}
\end{column}

\begin{column}{0.48\textwidth}
\textbf{Recovery ($\Delta$NDVI 2019–2025):}
\begin{itemize}
    \item No/Negative ($<$ 0)
    \item Low (0–0.1)
    \item Moderate (0.1–0.25)
    \item High ($<$ 0.25)
\end{itemize}
\end{column}
\end{columns}

\bigskip
\begin{block}{Spearman Correlation}
$\rho = 0.669$, $p < 0.001$ \\
\textbf{Strong positive relationship:} Higher severity $\rightarrow$ Higher recovery
\end{block}
\end{frame}

\begin{frame}{Severity × Recovery — Visual Analysis}
\centering
The positive correlation was also plotted to show it visually.

\includegraphics[width=0.95\textwidth]{Images/Severity_x_Recovery.png}
\captionof{figure}{\scriptsize Left: Recovery distribution by severity class. Right: Spatial recovery pattern.}
\end{frame}

%-----------------------------------------------------------
\section{Spatial Variability Analysis}

\begin{frame}[fragile]{Focal Standard Deviation Analysis}

Calculated on total NDVI recovery respect to the same season (Jul 2019 – Jul 2025)
\begin{itemize}
    \item Three window sizes: 3×3, 5×5, 7×7 pixels
    \item Spatial scales: 30m, 50m, 70m
    \item Focal SD Color scale: 99th percentile (0.014) to optimize contrast
\end{itemize}

\bigskip
\textbf{Palette Choice:}
\begin{itemize}
    \item Viridis: used to plot the Total recovery map
    \item Inverted cividis: used for the focal SD panels. 
    
    \textcolor{yellow!80!black}{Yellow} (low SD) $\rightarrow$ \textcolor{blue!80!black}{Blue} (high SD)
\end{itemize}
\end{frame}

\begin{frame}{Visual Comparison — Fine vs. Coarse Scale}
\centering
\includegraphics[width=0.7\textwidth]{Images/SD_Recovery_Variability.png}
\captionof{figure}{\scriptsize Focal SD at three spatial scales. High SD (blue) at edges and complex terrain.}
\end{frame}

%-----------------------------------------------------------
\section{Results and Discussion}

\begin{frame}{Results and Discussion — Key Findings}
\begin{itemize}
    \item After 6.5 years, mean NDVI remains \textbf{13\% below pre-fire baseline}, reflecting slow canopy maturation (decades required) and ongoing \textbf{vegetation-type transitions};
    
    \item The presence of a strong positive correlation ($\rho = 0.669$) between burn severity and recovery is ecologically meaningful. 

    \item High recovery $\neq$ "forest recovery". The initial fast recovery mainly represents shrubland/oak woodland establishment, and only later does the forest recovery develop and become predominant.
    
    \item Spatial analyses confirm that recovery shows heterogeneity.
\end{itemize}
\end{frame}

%-----------------------------------------------------------
\section{Conclusions}

\begin{frame}{Conclusions}
\begin{itemize}
    \item Camp Fire caused \textbf{severe NDVI decline} (mean dNDVI = 0.143, representing 47\% vegetation loss);
    
    \item By July 2025, \textbf{~64\% of lost productivity restored}, but mean NDVI remains 13\% below pre-fire levels;
    
    \item \textbf{Strong positive correlation} between burn severity and recovery was recorded ($\rho = 0.669$, $p < 0.001$);
    
    \item Recovery is \textbf{spatially heterogeneous}, particularly at finer scales (3×3);
    
    \item Recovery rate slowed from the equivalent of +0.075/yr in the first 8 months  to -0.002/yr in the period July 2024-2025, suggesting the start of a slower development;
    
    \item \textbf{Sentinel-2 imagery} proves effective for monitoring post-fire vegetation dynamics.
\end{itemize}
\end{frame}

%-----------------------------------------------------------
\section{References}

\begin{frame}{References}
\begin{itemize}
    \item \textbf{BLM Camp Fire Reforestation Plan} (2021): Climate-informed restoration strategies for Butte County, CA. \\
    \url{https://www.blm.gov/sites/default/files/docs/2021-09/BLM_CampPlan.pdf}
    
    \item \textbf{Monitoring Trends in Burn Severity (MTBS)}: \\
    \url{https://www.mtbs.gov/}
    
    \item \textbf{Copernicus Browser (Sentinel-2)}: \\
    \url{https://browser.dataspace.copernicus.eu/}
\end{itemize}
\end{frame}

%-----------------------------------------------------------
\begin{frame}
\centering
\huge {Thank you for your attention!}


\bigskip
\vfill
\footnotesize 
GitHub: \url{https://github.com/Serena Casagrande}

\end{frame}

\end{document}
